\chapter{\ifenglish Introduction\else บทนำ\fi}

\section{\ifenglish Project rationale\else ที่มาของโครงงาน\fi}
เนื่องจากห้องเรียน Active Learning ที่ศูนย์นวัตกรรมการสอนและการเรียนรู้ถูกออกแบบมาเพื่อทดลองการจัดการเรียนรู้รูปแบบใหม่
\enskip ที่เน้น Active Learning มีการจัดหาโต๊ะและเก้าอีกที่มีล้อเคลื่อนย้ายสะดวก จัดรูปแบบห้องได้หลากหลาย มีการดานที่สามารถเขียนได้รอบห้อง เหมาะกับการทำงานกลุ่ม
\enskip และต้องการที่จะศึกษารูปแบบการใช้งานห้องเรียนว่ามีแบบใดบ้าง และมีประสิทธิภาพอย่างไร
\section{\ifenglish Objectives\else วัตถุประสงค์ของโครงงาน\fi}
% \begin{enumerate}
%     \item
% \end{enumerate}
เนื่องจากการที่จะศึกษารูปแบบและลักษณะการใช้ห้องเรียน Active Learning ที่ศูนย์นวัตกรรมการสอนและการเรียนรู้จำเป็นจะต้องใช้การสังเกตการณ์จากมนุษย์
\enskip ซึ่งการสังเกตการณ์จากมนุษย์เป็นข้อมูลที่ต้องใช้ต้นทุนในการเก็บข้อมูลสูง
\enskip โครงงานนี้จึงได้ออกแบบระบบที่สามารถแยกแยะรูปแบบการจัดวางโต๊ะได้โดยอัตโนมัติ เพื่อใช้เป็นข้อมูลตั้งต้นในการระบุรูปแบบการใช้งานห้อง
\enskip สามารถเก็บข้อมูลได้ตลอดเวลา และสามารถนำข้อมูลการแยกแยะไปใช้ในการวิเคราะห์และประเมินผลต่อไปได้

\section{\ifenglish Project scope\else ขอบเขตของโครงงาน\fi}
    \begin{enumerate}
        \item ห้องเรียนที่ต้องการศึกษามีการจัดเรียงโต๊ะที่มองเห็นและแยกแยะเป็นรูปแบบที่ซ้ำๆกันได้
        \item มุมกล้องที่ใช้ในการถ่ายภาพต้องสามารถถ่ายภาพได้ทั้งห้อง และมีแสงเพียงพอที่จะสามารถเก็บรายละเอียดในห้องได้
    \end{enumerate}
\subsection{\ifenglish Hardware scope\else ขอบเขตด้านฮาร์ดแวร์\fi}
    \begin{enumerate}
        \item Raspberry Pi 4 และ 5MP OV5647 Fisheye Camera Module for Raspberry Pi
        \item PC, Mobile devices สำหรับการเข้าใช้งานระบบ
    \end{enumerate}
\subsection{\ifenglish Software scope\else ขอบเขตด้านซอฟต์แวร์\fi}
    ใช้ Raspberry OS และภาษา Python ในการสร้าง Application สำหรับการถ่ายภาพและส่งข้อมูลไปยัง Firebase และใช้ Keras DBSCAN Model ในการจำแนกแยกแยะรูปแบบการจัดเรียงโต๊ะ 
    \enskip และใช้ Flask ในการสร้างเว็บเพจแสดงผลข้อมูล
\section{\ifenglish Expected outcomes\else ประโยชน์ที่ได้รับ\fi}
    ลดต้นทุนในการเก็บข้อมูลโดยการสังเกตุการณ์ด้วยมนุษย์โดยการใช้เทคโนโลยี Neural Network ในการจำแนกแยกแยะรูปแบบการจัดเรียงโต๊ะ และสามารถนำข้อมูลไปใช้ในการวิเคราะห์และประเมินผลต่อไป
    \enskip และสามารถนำข้อมูลไปใช้ในการปรับปรุงรูปแบบการใช้ห้องเรียนให้มีประสิทธิภาพในการเรียนรู้มากขึ้น
\section{\ifenglish Technology and tools\else เทคโนโลยีและเครื่องมือที่ใช้\fi}
    \begin{enumerate}
        \item Raspberry Pi OS ในการทำหน้าที่สั่งการเก็บภาพผ่านกล้องถ่ายภาพผ่าน Library Picamera
        \item Google Firebase ในการเก็บข้อมูลภาพ
        \item Keras DBSCAN Model เป็น Convolution Neural Network ในการจำแนกองค์ประกอบของภาพและจำแนกกลุ่มภาพ
        \item Tensorflow ในการใช้ Hardware accleleration ในการทำงานของ Keras DBSCAN Model
        \item Flask ในการสร้างเว็บเพจแสดงผลข้อมูล
    \end{enumerate}
\subsection{\ifenglish Hardware technology\else เทคโนโลยีด้านฮาร์ดแวร์\fi}
    ขึ้นอยู่กับ Performance ของ CPU ของอุปกรณ์ที่ใช้ในการจำแนกภาพให้อยู่ในกลุ่มต่างๆว่าสามารถทำได้รวดเร็วมากน้อยเพียงใด
\subsection{\ifenglish Software technology\else เทคโนโลยีด้านซอฟต์แวร์\fi}
    \begin{itemize}
        \item Python
        \item Bash
        \item Google CLoud Platform
        \item Visual Studio Code
    \end{itemize}
\section{\ifenglish Project plan\else แผนการดำเนินงาน\fi}

\begin{plan}{6}{2023}{3}{2024}
    \planitem{6}{2023}{7}{2023}{ศึกษาค้นคว้า}
    \planitem{8}{2023}{10}{2023}{ทำระบบเก็บภาพ Dataset}
    \planitem{11}{2023}{2}{2024}{สร้าง WebApp จำแนกรูปภาพและแสดงผลข้อมูล}
    \planitem{8}{2023}{3}{2024}{ทดสอบ}
\end{plan}

\section{\ifenglish Roles and responsibilities\else บทบาทและความรับผิดชอบ\fi}
เริ่มต้นหาข้อมูลว่าการจำแนกกลุ่มรูปภาพสามารถใช้วิธีไดได้บ้าง และเริ่มทำการเก็บภาพเพื่อใช้ในการทดสอบวิธีการจำแนกแยกแยะรูปภาพ
\enskip ต่อมาจะทำการสร้าง WebApp ที่สามารถแสดงผลข้อมูลที่ได้จากการจำแนกแยกแยะรูปภาพ และทำการทดสอบระบบว่าสามารถทำงานได้ตามที่ต้องการหรือไม่

\section{\ifenglish%
Impacts of this project on society, health, safety, legal, and cultural issues
\else%
ผลกระทบด้านสังคม สุขภาพ ความปลอดภัย กฏหมาย และวัฒนธรรม
\fi}
มีผลกระทบด้านสังคมและความเป็นส่วนตัวของผู้ใช้งานห้องเรียนเนื่องจากมีการเก็บภาพที่อาจจะเป็นข้อมูลส่วนตัวของผู้ใช้งานและแต่ได้ใช้การลดความชัดเจนของภาพเพื่อไม่ให้เห็นข้อมูลส่วนตัวของผู้ใช้งานห้องเรียน
\section{แนวทางและโยชน์ในการประยุกต์ใช้งานโครงงานกับงานในด้านอื่นๆ รวมถึงผลกระทบในด้านสังคมและสิ่งแวดล้อมจากการใช้ความรู้ทางวิศวกรรมที่ได้}
โครงงานนี้สามารถนำไปใช้ในการจำแนกแยกแยะรูปแบบการจัดเรียงโต๊ะในห้องเรียนอื่นๆ ที่มีการจัดเรียงโต๊ะที่มีรูปแบบที่สามารถจำแนกเป็นกลุ่มได้
\enskip และสามารถนำข้อมูลไปใช้ในการปรับปรุงรูปแบบการใช้ห้องเรียนให้มีประสิทธิภาพในการเรียนรู้มากขึ้น
