\chapter{\ifproject%
\ifenglish Project Structure and Methodology\else โครงสร้างและขั้นตอนการทำงาน\fi
\else%
\ifenglish Project Structure\else โครงสร้างของโครงงาน\fi
\fi
}

ในบทนี้จะกล่าวถึงหลักการ และการออกแบบระบบ ที่ใช้ในการพัฒนาโครงงาน โดยจะแบ่งเป็น 3 ส่วน คือ
\enskip Frontend, Backend, Hardware

% \makeatletter

% \renewcommand\section{\@startsection {section}{1}{\z@}%
%                                    {13.5ex \@plus -1ex \@minus -.2ex}%
%                                    {2.3ex \@plus.2ex}%
%                                    {\normalfont\large\bfseries}}

% \makeatother
%\vspace{2ex}
% \titleformat{\section}{\normalfont\bfseries}{\thesection}{1em}{}
% \titlespacing*{\section}{0pt}{10ex}{0pt}

\section{Frontend}

ส่วนหน้าต่าง WebApp จะใช้ Flask ในการสร้างเว็บเพจแสดงผลข้อมูล และใช้ HTML และ CSS ในการจัดรูปแบบหน้าเว็บ

\section{Backend}

จะนำภาพจาก Firebase ที่มีการเก็บภาพของแต่ละวันที่มาการใช้ห้องเรียน Active Learning มาใช้ในการจำแนกแยกแยะรูปแบบการจัดเรียงโต๊ะ โดยใช้ Keras DBSCAN Model ในการจำแนกแยกแยะรูปแบบการจัดเรียงโต๊ะ
\enskip ออกมาเป็น Cluster จำแนกกลุ่ม Layout การจัดโต๊ะ ต่อมาก็ทำการจัดการ Cluster แต่ละกลุ่มเช่นการ เปลี่ยนชื่อกลุ่ม ลบกลุ่มที่คิดว่าเป็น Noise ออกเพื่อเวลาแสดงผลข้อมูลจะได้ไม่มีข้อมูลที่ไม่เกี่ยวข้อง
\enskip การรวมกลุ่มที่มองเห็นว่าสามารถมองได้เป็นกลุ่มเดียวกันแต่อาจจะถูกแยกออกมาเป็นกลุ่มต่างกัน และทำการแสดงผลข้อมูลที่ได้จากการจำแนกแยกแยะรูปแบบการจัดเรียงโต๊ะและรูปแบบการใช้ห้อง

\section{Hardware}
Hardware หลักจะเป็น Raspberry Pi Model 4 และ 5MP OV5647 Fisheye Camera Module for Raspberry Pi ในการถ่ายภาพและส่งข้อมูลไปยัง Firebase
\enskip ทั้งนี้ Firebase มีข้อจำกัดสำหรับผู้ใช้งานที่ไม่ได้ทำการ Subscription ในการใช้งาน โดยมีข้อจำกัดในด้านพื้นที่เก็บข้อมูลรูปภาพ จึงต้องออกแบบระบบที่หลีกเลี่ยงข้อจำกัดนี้
\enskip ด้วยวิธีออกแบบระบบที่มีการถ่ายภาพการใช้ห้องเรียนที่มีการบันทึกชื่อภาพเป็นวันที่และเวลา และมีการลบข้อมูลภาพที่เก่าออกไปเมื่อเกินเวลาที่กำหนดโดยอัตโนมัติ
\enskip นอกจากการลบข้อมูลภาพเก่าออกจาก Firebase ยังต้องทดสอบหาความถี่ในการถ่ายภาพที่เหมาะสมที่จะทำให้ได้ข้อมูลที่มีความถูกต้องและเพียงพอ โดยที่นี้จะตั้งค่าการถ่ายภาพทุก 5 นาที
\enskip ในช่วงแรกที่ทำการทดสอบ Model เมื่อคิดว่าตั้งค่า Model ได้ผลลัพธ์ที่ดีแล้ว จะทำการลดความถี่ในการถ่ายภาพลงเหลือทุก 10 นาทีแทนเพื่อลดต้นทุนพื้นที่ในด้านการเก็บข้อมูล
\enskip ต่อมาจะต้องตั้งต่าให้ Raspberry Pi ทำงานอยู่ในระบบเครือข่ายที่มีความเสถียรและมีความเร็วในการส่งข้อมูลที่เพียงพอ โดยที่นี้จะใช้เครือข่ายของ TLIC และสามารถที่จะเข้าไป Maintainance ได้ง่าย
\enskip ผ่าน Remote Desktop หรือ SSH ผ่าน VPN แต่เนื่องจากระบบเครือข่ายไม่อนุญาตให้ใช้ Static IP จึงต้องใช้ Software PiTunnel ในการเชื่อมต่อเข้าไปใน Raspberry Pi ผ่าน SSH โดยที่ไม่ต้องใช้ Static IP


% \begin{figure}
% \begin{center}
% \includegraphics{800px-Briny_Beach.jpg}
% \end{center}
% \caption[Poem]{The Walrus and the Carpenter}
% \label{fig:walrus}
% \end{figure}

% \subsection{The Black Kitten}
%   One thing was certain, that the WHITE kitten had had nothing to
% do with it:---it was the black kitten's fault entirely~\cite{aiw}.  For the
% white kitten had been having its face washed by the old cat for
% the last quarter of an hour (and bearing it pretty well,
% considering); so you see that it COULDN'T have had any hand in
% the mischief.

%   The way Dinah washed her children's faces was this:  first she
% held the poor thing down by its ear with one paw, and then with
% the other paw she rubbed its face all over, the wrong way,
% beginning at the nose:  and just now, as I said, she was hard at
% work on the white kitten, which was lying quite still and trying
% to purr---no doubt feeling that it was all meant for its good.

%   But the black kitten had been finished with earlier in the
% afternoon, and so, while Alice was sitting curled up in a corner
% of the great arm-chair, half talking to herself and half asleep,
% the kitten had been having a grand game of romps with the ball of
% worsted Alice had been trying to wind up, and had been rolling it
% up and down till it had all come undone again; and there it was,
% spread over the hearth-rug, all knots and tangles, with the
% kitten running after its own tail in the middle.

% \subsection{The Reproach}

%   `Oh, you wicked little thing!' cried Alice, catching up the
% kitten, and giving it a little kiss to make it understand that it
% was in disgrace.  `Really, Dinah ought to have taught you better
% manners!  You OUGHT, Dinah, you know you ought!' she added,
% looking reproachfully at the old cat, and speaking in as cross a
% voice as she could manage---and then she scrambled back into the
% arm-chair, taking the kitten and the worsted with her, and began
% winding up the ball again.  But she didn't get on very fast, as
% she was talking all the time, sometimes to the kitten, and
% sometimes to herself.  Kitty sat very demurely on her knee,
% pretending to watch the progress of the winding, and now and then
% putting out one paw and gently touching the ball, as if it would
% be glad to help, if it might.

%   `Do you know what to-morrow is, Kitty?' Alice began.  `You'd
% have guessed if you'd been up in the window with me---only Dinah
% was making you tidy, so you couldn't.  I was watching the boys
% getting in stick for the bonfire---and it wants plenty of
% sticks, Kitty!  Only it got so cold, and it snowed so, they had
% to leave off.  Never mind, Kitty, we'll go and see the bonfire
% to-morrow.'  Here Alice wound two or three turns of the worsted
% round the kitten's neck, just to see how it would look:  this led
% to a scramble, in which the ball rolled down upon the floor, and
% yards and yards of it got unwound again.

%   `Do you know, I was so angry, Kitty,' Alice went on as soon as
% they were comfortably settled again, `when I saw all the mischief
% you had been doing, I was very nearly opening the window, and
% putting you out into the snow!  And you'd have deserved it, you
% little mischievous darling!  What have you got to say for
% yourself?  Now don't interrupt me!' she went on, holding up one
% finger.  `I'm going to tell you all your faults.  Number one:
% you squeaked twice while Dinah was washing your face this
% morning.  Now you can't deny it, Kitty:  I heard you!  What that
% you say?' (pretending that the kitten was speaking.)  `Her paw
% went into your eye?  Well, that's YOUR fault, for keeping your
% eyes open---if you'd shut them tight up, it wouldn't have
% happened.  Now don't make any more excuses, but listen!  Number
% two:  you pulled Snowdrop away by the tail just as I had put down
% the saucer of milk before her!  What, you were thirsty, were you?
