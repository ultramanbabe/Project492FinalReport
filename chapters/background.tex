\chapter{\ifenglish Background Knowledge and Theory\else ทฤษฎีที่เกี่ยวข้อง\fi}

การทำโครงงาน เริ่มต้นด้วยการศึกษาค้นคว้า ทฤษฎีที่เกี่ยวข้องกับโครงงาน ที่เคยมีผู้นำเสนอไว้แล้ว
\enskip ซึ่งเนื้อหาในบทนี้ก็จะเกี่ยวกับการอธิบายถึงสิ่งที่เกี่ยวข้องกับโครงงาน เพื่อให้ผู้อ่านเข้าใจเนื้อหาในบทถัดๆ ไปได้ง่ายขึ้น

\section{Backend}
\subsection{Computer Vision}
Computer Vision เป็นเทคโนโลยีที่ใช้ในการประมวลผลภาพและวิดีโอ
\enskip เพื่อให้คอมพิวเตอร์สามารถเห็นและเข้าใจโลกได้ โดยใช้เทคนิคการประมวลผลภาพและวิดีโอ 
\enskip และการเรียนรู้ของเครื่อง ซึ่งสามารถนำไปใช้ในหลากหลายงาน เช่น การตรวจจับวัตถุ การจดจำใบหน้า การติดตามวัตถุ การวิเคราะห์ท่าทาง การจดจำเสียง และอื่นๆ \cite{website:computervision}

\subsection{Machine Learning}
Machine Learning คือการเรียนรู้และวิเคราห์ข้อมูลด้วย Algorithm โดยไม่ต้องใช้กฎหรือเงื่อนไขที่ถูกกำหนดไว้ล่วงหน้า เช่น การจำแนกภาพ การจดจำเสียง การทำนายข้อมูล และอื่นๆ
\enskip โดย Machine Learning แบ่งได้เป็น 3 แบบใหญ่ๆได้แก่ \cite{website:machinelearning}
\begin{enumerate}
    \item Supervised Learning คือการเรียนรู้จากข้อมูลที่ได้กำหนดผลลัพธ์ไว้ล่วงหน้าเช่น ภาพที่มี Label ว่าเป็นอะไร
    \item Unsupervised Learning คือการเรียนรู้จากข้อมูลที่ไม่ได้กำหนดผลลัพธ์ไว้ล่วงหน้าเช่น การจัดกลุ่มข้อมูลว่ามีลักษณะเดียวกันได้กี่กลุ่ม
    \item Reinforcement Learning คือการเรียนรู้จากการทำงานและการลองผิดลองถูก และทำการวัดผลลัพธ์ว่าดีหรือไม่ดีผ่านการให้คะแนนความแม่นยำของผลลัพธ์
\end{enumerate}

\subsection{MongoDB}

MongoDB เป็นฐานข้อมูลแบบ NoSQL ที่ใช้เก็บข้อมูลแบบ Document โดยไม่ต้องมีโครงสร้างของข้อมูลที่แน่นอนล่วงหน้า สามารถเก็บข้อมูลได้หลากหลายรูปแบบ และสามารถเพิ่มเติมข้อมูลได้ง่าย โดยไม่ต้องเปลี่ยนโครงสร้างของข้อมูลที่มีอยู่ \cite{website:mongodb}

\subsection{DBSCAN}

DBSCAN (Density-Based Spatial Clustering of Applications with Noise) เป็น Algorithm ในการจำแนกข้อมูลแบบคลัสเตอร์ โดยจะจำแนกข้อมูลที่มีความห่าง(ความแตกต่าง)ใกล้กันเป็นกลุ่มเดียวกัน
\enskip โดยจะจำแนกข้อมูลที่มีความห่างใกล้กันเป็นกลุ่มเดียวกัน และจำแนกข้อมูลที่มีความห่างห่างกันเป็นกลุ่มต่างกัน \cite{website:dbscan}

\subsection{Firebase}

Firebase เป็น Platform ที่ให้บริการ Backend สำหรับการพัฒนา Application โดยไม่ต้องสร้าง Server ขึ้นมาเอง สามารถใช้งานได้ฟรี และมีบริการหลากหลาย เช่น การเก็บข้อมูล การจัดการผู้ใช้ การจัดการการยืนยันตัวตน
\enskip โดยโครงงานนี้จะมุ่งเน้นไปที่การจัดเก็บข้อมูลรูปภาพที่ถ่ายห้องเรียนไว้ \cite{website:firebase}

\section{Frontend}

\subsection{Flask}

Flask เป็น Web Framework สำหรับการพัฒนา Web Application ด้วยภาษา Python โดยสามารถสร้าง Web Application ได้ง่าย และมีความยืดหยุ่นในการใช้งาน \cite{website:flask}

\subsection{HTML}

HTML (HyperText Markup Language) เป็นภาษาที่ใช้ในการสร้างเว็บเพจ โดยมี Element ต่างๆ ที่ใช้ในการสร้างเว็บเพจ เช่น การสร้าง Header การสร้าง Paragraph การสร้าง Table และอื่นๆ \cite{website:html}

\subsection{CSS}

CSS (Cascading Style Sheets) เป็นภาษาที่ใช้ในการจัดรูปแบบหน้าเว็บ เช่น การจัดรูปแบบข้อความ การจัดรูปแบบสีพื้นหลัง การจัดรูปแบบขอบ และอื่นๆ \cite{website:css}

\section{Hardware}

\subsection{Raspberry Pi}

Raspberry Pi เป็นคอมพิวเตอร์ขนาดเล็กที่สามารถใช้งานได้หลากหลาย เช่น ใช้เป็น Server ใช้เป็นเครื่องคอมพิวเตอร์ส่วนตัว ใช้เป็นเครื่องควบคุมอุปกรณ์อิเล็กทรอนิกส์
\enskip โดยในโครงงานนี้ใช้ Raspberry Pi ในการถ่ายภาพและส่งข้อมูลไปยัง Firebase และนำภาพที่ถ่ายมาประมวลและแสดงผลต่อ \cite{website:raspberrypi}

% \subsection{Google Cloud Platform}

% Google Cloud Platform เป็น Platform ที่ให้บริการ Cloud Computing โดยมีบริการหลากหลาย เช่น การเก็บข้อมูล การจัดการ Server การจัดการ Network
% \enskip แต่โครงงานนี้ใช้เป็นเครื่องมือจัดการเชื่อมต่อ Firebase กับ Database และ Flask กับ Server \cite{website:googlecloudplatform}



% \subsubsection{Subsubsection 1 heading goes here}
% Subsubsection 1 text

% \subsubsection{Subsubsection 2 heading goes here}
% Subsubsection 2 text

% \section{Third section}
% Section 3 text. The dielectric constant\index{dielectric constant}
% at the air-metal interface determines
% the resonance shift\index{resonance shift} as absorption or capture occurs
% is shown in Equation~\eqref{eq:dielectric}:

% \begin{equation}\label{eq:dielectric}
% k_1=\frac{\omega}{c({1/\varepsilon_m + 1/\varepsilon_i})^{1/2}}=k_2=\frac{\omega
% \sin(\theta)\varepsilon_\mathit{air}^{1/2}}{c}
% \end{equation}

% \noindent
% where $\omega$ is the frequency of the plasmon, $c$ is the speed of
% light, $\varepsilon_m$ is the dielectric constant of the metal,
% $\varepsilon_i$ is the dielectric constant of neighboring insulator,
% and $\varepsilon_\mathit{air}$ is the dielectric constant of air.

% \section{About using figures in your report}

% define a command that produces some filler text, the lorem ipsum.
% \newcommand{\loremipsum}{
%   \textit{Lorem ipsum dolor sit amet, consectetur adipisicing elit, sed do
%   eiusmod tempor incididunt ut labore et dolore magna aliqua. Ut enim ad
%   minim veniam, quis nostrud exercitation ullamco laboris nisi ut
%   aliquip ex ea commodo consequat. Duis aute irure dolor in
%   reprehenderit in voluptate velit esse cillum dolore eu fugiat nulla
%   pariatur. Excepteur sint occaecat cupidatat non proident, sunt in
%   culpa qui officia deserunt mollit anim id est laborum.}\par}

% \begin{figure}
%   \centering

%   \fbox{
%      \parbox{.6\textwidth}{\loremipsum}
%   }

%   % To include an image in the figure, say myimage.pdf, you could use
%   % the following code. Look up the documentation for the package
%   % graphicx for more information.
%   % \includegraphics[width=\textwidth]{myimage}

%   \caption[Sample figure]{This figure is a sample containing \gls{lorem ipsum},
%   showing you how you can include figures and glossary in your report.
%   You can specify a shorter caption that will appear in the List of Figures.}
%   \label{fig:sample-figure}
% \end{figure}

% Using \verb.\label. and \verb.\ref. commands allows us to refer to
% figures easily. If we can refer to Figures
% \ref{fig:walrus} and \ref{fig:sample-figure} by name in the {\LaTeX}
% source code, then we will not need to update the code that refers to it
% even if the placement or ordering of the figures changes.

% \loremipsum\loremipsum

% This code demonstrates how to get a landscape table or figure. It
% uses the package lscape to turn everything but the page number into
% landscape orientation. Everything should be included within an
% \afterpage{ .... } to avoid causing a page break too early.
% \afterpage{
%   \begin{landscape}
%   \begin{table}
%     \caption{Sample landscape table}
%     \label{tab:sample-table}

%     \centering

%     \begin{tabular}{c||c|c}
%         Year & A & B \\
%         \hline\hline
%         1989 & 12 & 23 \\
%         1990 & 4 & 9 \\
%         1991 & 3 & 6 \\
%     \end{tabular}
%   \end{table}
%   \end{landscape}
% }

% \loremipsum\loremipsum\loremipsum

% \section{Overfull hbox}

% When the \verb.semifinal. option is passed to the \verb.cpecmu. document class,
% any line that is longer than the line width, i.e., an overfull hbox, will be
% highlighted with a black solid rule:
% \begin{center}
% \begin{minipage}{2em}
% juxtaposition
% \end{minipage}
% \end{center}

\section{\ifenglish%
\ifcpe CPE \else ISNE \fi knowledge used, applied, or integrated in this project
\else%
ความรู้ตามหลักสูตรซึ่งถูกนำมาใช้หรือบูรณาการในโครงงาน
\fi
}

ได้ใช้ความรู้ด้านการเขียนหน้าเวปไซต์โดยใช้ HTML ร่วมกับ Flask ในการสร้างเวปเพจแสดงผลข้อมูล ต่อมาได้ใช้ความรู้เกี่ยวกับการจัดเก็บข้อมูลจากวิชา Database ในการจัดเก็บข้อมูลรูปภาพที่ถ่ายไว้
\enskip และใช้ความรู้ด้านการประมวลผลภาพจากวิชา Computer Vision ในการจำแนกแยกแยะรูปภาพ และใช้ความรู้ในการใช้ Micro Controller ในการควบคุม Raspberry Pi ในการถ่ายภาพและส่งข้อมูลไปยัง
\enskip ใช้ความรู้เกี่ยวกับ Cloud Platform ในการเก็บข้อมูลรูปภาพไว้ใน Firebase และการ Deploy Web Application 

\section{\ifenglish%
Extracurricular knowledge used, applied, or integrated in this project
\else%
ความรู้นอกหลักสูตรซึ่งถูกนำมาใช้หรือบูรณาการในโครงงาน
\fi
}

การเลือกใช้อุปกรณ์ Micro Controller ที่เหมาะสมกับการใช้งานตามที่ต้องการและได้ผลลัพธ์เป็น Raspberry Pi การหาข้อมูลเพิ่มเติมเกี่ยวกับการใช้ภาษา Python ในการสร้างแอพพลิเคชัน
\enskip ศึกษาการใช้ Machine Learning ในการจำแนกแยกแยะรูปภาพและหาข้อมูลเพิ่มเติมเกี่ยวกับการใช้ Keras DBSCAN Model ที่มีความเหมาะสมกับการทำงานโครงงานนี้
\enskip การใช้ Git ในการจัดการโครงงานและควบคุม Stage ของโครงงานเผื่อเกิดความผิดพลาดในการพัฒนาโครงงานก็จะสามารถย้อนกลับไปแก้ไขได้