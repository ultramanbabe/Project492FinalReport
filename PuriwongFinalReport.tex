\documentclass[semifinal]{cpecmu}

%% This is a sample document demonstrating how to use the CPECMU
%% project template. If you are having trouble, see "cpecmu.pdf" for
%% documentation.

\projectNo{P069-1}
\acadyear{2021}

\titleTH{โครงงานสุดเลิฟของฉัน}
\titleEN{Your Project Name Goes Here}

\author{นายกินรี ไทร์ล้ำเลิศ}{Kinnaree Tirelumlert}{690610696}
\author{นายบรรจบ พบเอฟตลอด}{Banjob Pob-eftalord}{690610969}

\cpeadvisor{chinawat}
\cpecommittee{paskorn}
\committee{รศ.ดร.\,นิพนธ์ ธีรอำพน}{Assoc.\,Prof.\,Nipon Theera-Umpon, Ph.D.}

%% Some possible packages to include:
\usepackage[final]{graphicx} % for including graphics

%% Add bookmarks and hyperlinks in the document.
\PassOptionsToPackage{hyphens}{url}
\usepackage[colorlinks=true,allcolors=Blue4,citecolor=red,linktoc=all]{hyperref}
\def\UrlLeft#1\UrlRight{$#1$}

%% Needed just by this example, but maybe not by most reports
\usepackage{afterpage} % for outputting
\usepackage{pdflscape} % for landscape figures and tables. 

%% Some other useful packages. Look these up to find out how to use
%% them.
% \usepackage{natbib}    % for author-year citation styles
% \usepackage{txfonts}
% \usepackage{appendix}  % for appendices on a per-chapter basis
% \usepackage{xtab}      % for tables that go over multiple pages
% \usepackage{subfigure} % for subfigures within a figure
% \usepackage{pstricks,pdftricks} % for access to special PostScript and PDF commands
% \usepackage{nomencl}   % if you have a list of abbreviations

%% if you're having problems with overfull boxes, you may need to increase
%% the tolerance to 9999
% \tolerance=9999

\bibliographystyle{plain}
% \bibliographystyle{IEEEbib}

% \renewcommand{\topfraction}{0.85}
% \renewcommand{\textfraction}{0.1}
% \renewcommand{\floatpagefraction}{0.75}

%% Example for glossary entry
%% Need to use glossary option
%% See glossaries package for complete documentation.
\ifglossary
  \newglossaryentry{lorem ipsum}{
    name=lorem ipsum,
    description={derived from Latin dolorem ipsum, translated as ``pain itself''}
  }
\fi

%% Uncomment this command to preview only specified LaTeX file(s)
%% imported with \include command below.
%% Any other file imported via \include but not specified here will not
%% be previewed.
%% Useful if your report is large, as you might not want to build
%% the entire file when editing a certain part of your report.
% \includeonly{chapters/intro,chapters/background}

\begin{document}
\maketitle
\makesignature

\ifproject
\begin{abstractTH}
การเขียนรายงานเป็นส่วนหนึ่งของการทำโครงงานวิศวกรรมคอมพิวเตอร์
เพื่อทบทวนทฤษฎีที่เกี่ยวข้อง อธิบายขั้นตอนวิธีแก้ปัญหาเชิงวิศวกรรม และวิเคราะห์และสรุปผลการทดลองอุปกรณ์และระบบต่างๆ
\enskip โดยต้องการที่จะสร้างระบบที่สามารถสำรวจและเก็บข้อมูลของสถานที่ต่างๆ โดยใช้เทคโนโลยีเซนเซอร์การจัดเก็บภาพนิ่ง
\enskip และการเชื่อมต่อเข้ากับเว็บไซต์เพื่อให้ผู้ใช้งานสามารถเข้าถึงข้อมูลได้ง่ายขึ้น และสามารถแสดงข้อมูลในรูปแบบกราฟและกลุ่มรูปภาพ
\enskip เพื่อจัดเก็บข้อมูลไปพัฒนาการใช้สถานที่ได้อย่างมีประสิทธิภาพมากขึ้นในอนาคต
\end{abstractTH}

\begin{abstract}
Writing a report is part of doing a computer engineering project to review related theories, 
\enskip explain engineering problem-solving steps, and analyze and summarize the results of experiments on various devices and systems.
\enskip We aim to create a system that can explore and collect data from various places using still image sensor technology and connect to a website to make it easier for users to access data.
\enskip It can display data in the form of graphs and image groups to collect data for more efficient future use of places.

% Make sure your abstract sits inside the \texttt{abstract} environment.
\end{abstract}

\iffalse
\begin{dedication}
This document is dedicated to all Chiang Mai University students.

Dedication page is optional.
\end{dedication}
\fi % \iffalse

\begin{acknowledgments}
การทำโครงงานนี้เป็นผลจากความร่วมมือของหลายๆคน ขอขอบคุณอาจารย์ที่ปรึกษา และเพื่อนๆ ที่ได้ช่วยเหลือให้ความรู้ต่างๆ ที่เป็นประโยชน์ต่อการทำโครงงานนี้
\enskip รวมถึงเทคโนโลยีตัวช่วยต่างๆ ที่ได้มีส่วนช่วยในการทำโครงงานนี้
\enskip และการแนะนำข้อควรปรับปรุงต่างๆ ที่ทำให้โครงงานนี้สมบูรณ์และมีคุณภาพมากขึ้น
\enskip จนกระทั่งโครงงานนี้สามารถสร้างความสำเร็จให้กับผู้ที่เกี่ยวข้องกับโครงงานนี้
\enskip สุดท้ายนี้ผู้จัดทำหวังว่างานวิจัยฉบับนี้จะเป็นประโยชน์ต่อผู้ที่สนใจในงานวิจัยด้านนี้ที่สนใจศึกษาต่อไป 

\acksign{2024}{3}{28}
\end{acknowledgments}%
\fi % \ifproject

\contentspage

\ifproject
\figurelistpage

\tablelistpage
\fi % \ifproject

% \abbrlist % this page is optional

% \symlist % this page is optional

% \preface % this section is optional


\pagestyle{empty}\cleardoublepage
\normalspacing \setcounter{page}{1} \pagenumbering{arabic} \pagestyle{cpecmu}

\chapter{\ifenglish Introduction\else บทนำ\fi}

\section{\ifenglish Project rationale\else ที่มาของโครงงาน\fi}
เนื่องจากห้องเรียน Active Learning ที่ศูนย์นวัตกรรมการสอนและการเรียนรู้ถูกออกแบบมาเพื่อทดลองการจัดการเรียนรู้รูปแบบใหม่
\enskip ที่เน้น Active Learning มีการจัดหาโต๊ะและเก้าอีกที่มีล้อเคลื่อนย้ายสะดวก จัดรูปแบบห้องได้หลากหลาย มีการดานที่สามารถเขียนได้รอบห้อง เหมาะกับการทำงานกลุ่ม
\enskip และต้องการที่จะศึกษารูปแบบการใช้งานห้องเรียนว่ามีแบบใดบ้าง และมีประสิทธิภาพอย่างไร
\section{\ifenglish Objectives\else วัตถุประสงค์ของโครงงาน\fi}
% \begin{enumerate}
%     \item
% \end{enumerate}
เนื่องจากการที่จะศึกษารูปแบบและลักษณะการใช้ห้องเรียน Active Learning ที่ศูนย์นวัตกรรมการสอนและการเรียนรู้จำเป็นจะต้องใช้การสังเกตการณ์จากมนุษย์
\enskip ซึ่งการสังเกตการณ์จากมนุษย์เป็นข้อมูลที่ต้องใช้ต้นทุนในการเก็บข้อมูลสูง
\enskip โครงงานนี้จึงได้ออกแบบระบบที่สามารถแยกแยะรูปแบบการจัดวางโต๊ะได้โดยอัตโนมัติ เพื่อใช้เป็นข้อมูลตั้งต้นในการระบุรูปแบบการใช้งานห้อง
\enskip สามารถเก็บข้อมูลได้ตลอดเวลา และสามารถนำข้อมูลการแยกแยะไปใช้ในการวิเคราะห์และประเมินผลต่อไปได้

\section{\ifenglish Project scope\else ขอบเขตของโครงงาน\fi}
    \begin{enumerate}
        \item ห้องเรียนที่ต้องการศึกษามีการจัดเรียงโต๊ะที่มองเห็นและแยกแยะเป็นรูปแบบที่ซ้ำๆกันได้
        \item มุมกล้องที่ใช้ในการถ่ายภาพต้องสามารถถ่ายภาพได้ทั้งห้อง 
        \enskip และมีแสงเพียงพอที่จะสามารถเก็บรายละเอียดในห้องได้
    \end{enumerate}
\subsection{\ifenglish Hardware scope\else ขอบเขตด้านฮาร์ดแวร์\fi}
    \begin{enumerate}
        \item Raspberry Pi 4 และ 5MP OV5647 Fisheye Camera Module for Raspberry Pi
        \item PC, Mobile devices สำหรับการเข้าใช้งานระบบ
    \end{enumerate}
\subsection{\ifenglish Software scope\else ขอบเขตด้านซอฟต์แวร์\fi}
    ใช้ Raspberry OS และภาษา Python ในการสร้าง Application สำหรับการถ่ายภาพและส่งข้อมูลไปยัง Firebase และใช้ Keras DBSCAN Model ในการจำแนกแยกแยะรูปแบบการจัดเรียงโต๊ะ 
    \enskip และใช้ Flask ในการสร้างเว็บเพจแสดงผลข้อมูล
\section{\ifenglish Expected outcomes\else ประโยชน์ที่ได้รับ\fi}
    ลดต้นทุนในการเก็บข้อมูลโดยการสังเกตุการณ์ด้วยมนุษย์โดยการใช้เทคโนโลยี Neural Network ในการจำแนกแยกแยะรูปแบบการจัดเรียงโต๊ะ และสามารถนำข้อมูลไปใช้ในการวิเคราะห์และประเมินผลต่อไป
    \enskip และสามารถนำข้อมูลไปใช้ในการปรับปรุงรูปแบบการใช้ห้องเรียนให้มีประสิทธิภาพในการเรียนรู้มากขึ้น
\section{\ifenglish Technology and tools\else เทคโนโลยีและเครื่องมือที่ใช้\fi}
    \begin{enumerate}
        \item Raspberry Pi OS ในการทำหน้าที่สั่งการเก็บภาพผ่านกล้องถ่ายภาพผ่าน Library Picamera
        \item Google Firebase ในการเก็บข้อมูลภาพ
        \item Keras DBSCAN Model เป็น Convolution Neural Network ในการจำแนกองค์ประกอบของภาพและจำแนกกลุ่มภาพ
        \item Tensorflow ในการใช้ Hardware accleleration ในการทำงานของ Keras DBSCAN Model
        \item Flask ในการสร้างเว็บเพจแสดงผลข้อมูล
    \end{enumerate}
\subsection{\ifenglish Hardware technology\else เทคโนโลยีด้านฮาร์ดแวร์\fi}
    ขึ้นอยู่กับ Performance ของ CPU ของอุปกรณ์ที่ใช้ในการจำแนกภาพให้อยู่ในกลุ่มต่างๆว่าสามารถทำได้รวดเร็วมากน้อยเพียงใด
\subsection{\ifenglish Software technology\else เทคโนโลยีด้านซอฟต์แวร์\fi}
    \begin{itemize}
        \item Python
        \item Bash
        \item Google CLoud Platform
        \item MongoDB
        \item Visual Studio Code
    \end{itemize}
\section{\ifenglish Project plan\else แผนการดำเนินงาน\fi}

\begin{plan}{6}{2023}{3}{2024}
    \planitem{6}{2023}{7}{2023}{ศึกษาค้นคว้า}
    \planitem{8}{2023}{10}{2023}{ทำระบบเก็บภาพ Dataset}
    \planitem{11}{2023}{2}{2024}{สร้าง WebApp จำแนกรูปภาพและแสดงผลข้อมูล}
    \planitem{8}{2023}{3}{2024}{ทดสอบ}
\end{plan}

\section{\ifenglish Roles and responsibilities\else บทบาทและความรับผิดชอบ\fi}
เริ่มต้นหาข้อมูลว่าการจำแนกกลุ่มรูปภาพสามารถใช้วิธีไดได้บ้าง
\enskip และเริ่มทำการเก็บภาพเพื่อใช้ในการทดสอบวิธีการจำแนกแยกแยะรูปภาพ
\enskip ต่อมาจะทำการสร้าง WebApp ที่สามารถแสดงผลข้อมูลที่ได้จากการจำแนกแยกแยะรูปภาพ และทำการทดสอบระบบว่าสามารถทำงานได้ตามที่ต้องการหรือไม่

\section{\ifenglish%
Impacts of this project on society, health, safety, legal, and cultural issues
\else%
ผลกระทบด้านสังคม สุขภาพ ความปลอดภัย กฏหมาย และวัฒนธรรม
\fi}
มีผลกระทบด้านสังคมและความเป็นส่วนตัวของผู้ใช้งานห้องเรียนเนื่องจากมีการเก็บภาพที่อาจจะเป็นข้อมูลส่วนตัวของผู้ใช้งานและแต่ได้ใช้การลดความชัดเจนของภาพเพื่อไม่ให้เห็นข้อมูลส่วนตัวของผู้ใช้งานห้องเรียน
\section{แนวทางและโยชน์ในการประยุกต์ใช้งานโครงงานกับงานในด้านอื่นๆรวมถึงผลกระทบในด้านสังคมและสิ่งแวดล้อมจากการใช้ความรู้ทางวิศวกรรมที่ได้}
โครงงานนี้สามารถนำไปใช้ในการจำแนกแยกแยะรูปแบบการจัดเรียงโต๊ะในห้องเรียนอื่นๆ ที่มีการจัดเรียงโต๊ะที่มีรูปแบบที่สามารถจำแนกเป็นกลุ่มได้
\enskip และสามารถนำข้อมูลไปใช้ในการปรับปรุงรูปแบบการใช้ห้องเรียนให้มีประสิทธิภาพในการเรียนรู้มากขึ้น

\chapter{\ifenglish Background Knowledge and Theory\else ทฤษฎีที่เกี่ยวข้อง\fi}

การทำโครงงาน เริ่มต้นด้วยการศึกษาค้นคว้า ทฤษฎีที่เกี่ยวข้อง หรือ งานวิจัย/โครงงาน ที่เคยมีผู้นำเสนอไว้แล้ว ซึ่งเนื้อหาในบทนี้ก็จะเกี่ยวกับการอธิบายถึงสิ่งที่เกี่ยวข้องกับโครงงาน เพื่อให้ผู้อ่านเข้าใจเนื้อหาในบทถัดๆ ไปได้ง่ายขึ้น

\section{The first section}
The text for Section 1 goes here.

\section{Second section}
Section 2 text.

\subsection{Subsection heading goes here}

Subsection 1 text

\subsubsection{Subsubsection 1 heading goes here}
Subsubsection 1 text

\subsubsection{Subsubsection 2 heading goes here}
Subsubsection 2 text

\section{Third section}
Section 3 text. The dielectric constant\index{dielectric constant}
at the air-metal interface determines
the resonance shift\index{resonance shift} as absorption or capture occurs
is shown in Equation~\eqref{eq:dielectric}:

\begin{equation}\label{eq:dielectric}
k_1=\frac{\omega}{c({1/\varepsilon_m + 1/\varepsilon_i})^{1/2}}=k_2=\frac{\omega
\sin(\theta)\varepsilon_\mathit{air}^{1/2}}{c}
\end{equation}

\noindent
where $\omega$ is the frequency of the plasmon, $c$ is the speed of
light, $\varepsilon_m$ is the dielectric constant of the metal,
$\varepsilon_i$ is the dielectric constant of neighboring insulator,
and $\varepsilon_\mathit{air}$ is the dielectric constant of air.

\section{About using figures in your report}

% define a command that produces some filler text, the lorem ipsum.
\newcommand{\loremipsum}{
  \textit{Lorem ipsum dolor sit amet, consectetur adipisicing elit, sed do
  eiusmod tempor incididunt ut labore et dolore magna aliqua. Ut enim ad
  minim veniam, quis nostrud exercitation ullamco laboris nisi ut
  aliquip ex ea commodo consequat. Duis aute irure dolor in
  reprehenderit in voluptate velit esse cillum dolore eu fugiat nulla
  pariatur. Excepteur sint occaecat cupidatat non proident, sunt in
  culpa qui officia deserunt mollit anim id est laborum.}\par}

\begin{figure}
  \centering

  \fbox{
     \parbox{.6\textwidth}{\loremipsum}
  }

  % To include an image in the figure, say myimage.pdf, you could use
  % the following code. Look up the documentation for the package
  % graphicx for more information.
  % \includegraphics[width=\textwidth]{myimage}

  \caption[Sample figure]{This figure is a sample containing \gls{lorem ipsum},
  showing you how you can include figures and glossary in your report.
  You can specify a shorter caption that will appear in the List of Figures.}
  \label{fig:sample-figure}
\end{figure}

Using \verb.\label. and \verb.\ref. commands allows us to refer to
figures easily. If we can refer to Figures
\ref{fig:walrus} and \ref{fig:sample-figure} by name in the {\LaTeX}
source code, then we will not need to update the code that refers to it
even if the placement or ordering of the figures changes.

\loremipsum\loremipsum

% This code demonstrates how to get a landscape table or figure. It
% uses the package lscape to turn everything but the page number into
% landscape orientation. Everything should be included within an
% \afterpage{ .... } to avoid causing a page break too early.
\afterpage{
  \begin{landscape}
  \begin{table}
    \caption{Sample landscape table}
    \label{tab:sample-table}

    \centering

    \begin{tabular}{c||c|c}
        Year & A & B \\
        \hline\hline
        1989 & 12 & 23 \\
        1990 & 4 & 9 \\
        1991 & 3 & 6 \\
    \end{tabular}
  \end{table}
  \end{landscape}
}

\loremipsum\loremipsum\loremipsum

\section{Overfull hbox}

When the \verb.semifinal. option is passed to the \verb.cpecmu. document class,
any line that is longer than the line width, i.e., an overfull hbox, will be
highlighted with a black solid rule:
\begin{center}
\begin{minipage}{2em}
juxtaposition
\end{minipage}
\end{center}

\section{\ifenglish%
\ifcpe CPE \else ISNE \fi knowledge used, applied, or integrated in this project
\else%
ความรู้ตามหลักสูตรซึ่งถูกนำมาใช้หรือบูรณาการในโครงงาน
\fi
}

อธิบายถึงความรู้ และแนวทางการนำความรู้ต่างๆ ที่ได้เรียนตามหลักสูตร ซึ่งถูกนำมาใช้ในโครงงาน

\section{\ifenglish%
Extracurricular knowledge used, applied, or integrated in this project
\else%
ความรู้นอกหลักสูตรซึ่งถูกนำมาใช้หรือบูรณาการในโครงงาน
\fi
}

อธิบายถึงความรู้ต่างๆ ที่เรียนรู้ด้วยตนเอง และแนวทางการนำความรู้เหล่านั้นมาใช้ในโครงงาน

\chapter{\ifproject%
\ifenglish Project Structure and Methodology\else โครงสร้างและขั้นตอนการทำงาน\fi
\else%
\ifenglish Project Structure\else โครงสร้างของโครงงาน\fi
\fi
}

ในบทนี้จะกล่าวถึงหลักการ และการออกแบบระบบ ที่ใช้ในการพัฒนาโครงงาน โดยจะแบ่งเป็น 3 ส่วน คือ
\enskip Frontend, Backend, Hardware

% \makeatletter

% \renewcommand\section{\@startsection {section}{1}{\z@}%
%                                    {13.5ex \@plus -1ex \@minus -.2ex}%
%                                    {2.3ex \@plus.2ex}%
%                                    {\normalfont\large\bfseries}}

% \makeatother
%\vspace{2ex}
% \titleformat{\section}{\normalfont\bfseries}{\thesection}{1em}{}
% \titlespacing*{\section}{0pt}{10ex}{0pt}

\section{Frontend}

ส่วนหน้าต่าง WebApp จะใช้ Flask ในการสร้างเว็บเพจแสดงผลข้อมูล และใช้ HTML และ CSS ในการจัดรูปแบบหน้าเว็บ

\section{Backend}

จะนำภาพจาก Firebase ที่มีการเก็บภาพของแต่ละวันที่มาการใช้ห้องเรียน Active Learning มาใช้ในการจำแนกแยกแยะรูปแบบการจัดเรียงโต๊ะ โดยใช้ Keras DBSCAN Model ในการจำแนกแยกแยะรูปแบบการจัดเรียงโต๊ะ
\enskip ออกมาเป็น Cluster จำแนกกลุ่ม Layout การจัดโต๊ะ ต่อมาก็ทำการจัดการ Cluster แต่ละกลุ่มเช่นการ เปลี่ยนชื่อกลุ่ม ลบกลุ่มที่คิดว่าเป็น Noise ออกเพื่อเวลาแสดงผลข้อมูลจะได้ไม่มีข้อมูลที่ไม่เกี่ยวข้อง
\enskip การรวมกลุ่มที่มองเห็นว่าสามารถมองได้เป็นกลุ่มเดียวกันแต่อาจจะถูกแยกออกมาเป็นกลุ่มต่างกัน และทำการแสดงผลข้อมูลที่ได้จากการจำแนกแยกแยะรูปแบบการจัดเรียงโต๊ะและรูปแบบการใช้ห้อง

\section{Hardware}
Hardware หลักจะเป็น Raspberry Pi Model 4 และ 5MP OV5647 Fisheye Camera Module for Raspberry Pi ในการถ่ายภาพและส่งข้อมูลไปยัง Firebase
\enskip ทั้งนี้ Firebase มีข้อจำกัดสำหรับผู้ใช้งานที่ไม่ได้ทำการ Subscription ในการใช้งาน โดยมีข้อจำกัดในด้านพื้นที่เก็บข้อมูลรูปภาพ จึงต้องออกแบบระบบที่หลีกเลี่ยงข้อจำกัดนี้
\enskip ด้วยวิธีออกแบบระบบที่มีการถ่ายภาพการใช้ห้องเรียนที่มีการบันทึกชื่อภาพเป็นวันที่และเวลา และมีการลบข้อมูลภาพที่เก่าออกไปเมื่อเกินเวลาที่กำหนดโดยอัตโนมัติ
\enskip นอกจากการลบข้อมูลภาพเก่าออกจาก Firebase ยังต้องทดสอบหาความถี่ในการถ่ายภาพที่เหมาะสมที่จะทำให้ได้ข้อมูลที่มีความถูกต้องและเพียงพอ โดยที่นี้จะตั้งค่าการถ่ายภาพทุก 5 นาที
\enskip ในช่วงแรกที่ทำการทดสอบ Model เมื่อคิดว่าตั้งค่า Model ได้ผลลัพธ์ที่ดีแล้ว จะทำการลดความถี่ในการถ่ายภาพลงเหลือทุก 10 นาทีแทนเพื่อลดต้นทุนพื้นที่ในด้านการเก็บข้อมูล
\enskip ต่อมาจะต้องตั้งต่าให้ Raspberry Pi ทำงานอยู่ในระบบเครือข่ายที่มีความเสถียรและมีความเร็วในการส่งข้อมูลที่เพียงพอ โดยที่นี้จะใช้เครือข่ายของ TLIC และสามารถที่จะเข้าไป Maintainance ได้ง่าย
\enskip ผ่าน Remote Desktop หรือ SSH ผ่าน VPN แต่เนื่องจากระบบเครือข่ายไม่อนุญาตให้ใช้ Static IP จึงต้องใช้ Software PiTunnel ในการเชื่อมต่อเข้าไปใน Raspberry Pi ผ่าน SSH โดยที่ไม่ต้องใช้ Static IP


% \begin{figure}
% \begin{center}
% \includegraphics{800px-Briny_Beach.jpg}
% \end{center}
% \caption[Poem]{The Walrus and the Carpenter}
% \label{fig:walrus}
% \end{figure}

% \subsection{The Black Kitten}
%   One thing was certain, that the WHITE kitten had had nothing to
% do with it:---it was the black kitten's fault entirely~\cite{aiw}.  For the
% white kitten had been having its face washed by the old cat for
% the last quarter of an hour (and bearing it pretty well,
% considering); so you see that it COULDN'T have had any hand in
% the mischief.

%   The way Dinah washed her children's faces was this:  first she
% held the poor thing down by its ear with one paw, and then with
% the other paw she rubbed its face all over, the wrong way,
% beginning at the nose:  and just now, as I said, she was hard at
% work on the white kitten, which was lying quite still and trying
% to purr---no doubt feeling that it was all meant for its good.

%   But the black kitten had been finished with earlier in the
% afternoon, and so, while Alice was sitting curled up in a corner
% of the great arm-chair, half talking to herself and half asleep,
% the kitten had been having a grand game of romps with the ball of
% worsted Alice had been trying to wind up, and had been rolling it
% up and down till it had all come undone again; and there it was,
% spread over the hearth-rug, all knots and tangles, with the
% kitten running after its own tail in the middle.

% \subsection{The Reproach}

%   `Oh, you wicked little thing!' cried Alice, catching up the
% kitten, and giving it a little kiss to make it understand that it
% was in disgrace.  `Really, Dinah ought to have taught you better
% manners!  You OUGHT, Dinah, you know you ought!' she added,
% looking reproachfully at the old cat, and speaking in as cross a
% voice as she could manage---and then she scrambled back into the
% arm-chair, taking the kitten and the worsted with her, and began
% winding up the ball again.  But she didn't get on very fast, as
% she was talking all the time, sometimes to the kitten, and
% sometimes to herself.  Kitty sat very demurely on her knee,
% pretending to watch the progress of the winding, and now and then
% putting out one paw and gently touching the ball, as if it would
% be glad to help, if it might.

%   `Do you know what to-morrow is, Kitty?' Alice began.  `You'd
% have guessed if you'd been up in the window with me---only Dinah
% was making you tidy, so you couldn't.  I was watching the boys
% getting in stick for the bonfire---and it wants plenty of
% sticks, Kitty!  Only it got so cold, and it snowed so, they had
% to leave off.  Never mind, Kitty, we'll go and see the bonfire
% to-morrow.'  Here Alice wound two or three turns of the worsted
% round the kitten's neck, just to see how it would look:  this led
% to a scramble, in which the ball rolled down upon the floor, and
% yards and yards of it got unwound again.

%   `Do you know, I was so angry, Kitty,' Alice went on as soon as
% they were comfortably settled again, `when I saw all the mischief
% you had been doing, I was very nearly opening the window, and
% putting you out into the snow!  And you'd have deserved it, you
% little mischievous darling!  What have you got to say for
% yourself?  Now don't interrupt me!' she went on, holding up one
% finger.  `I'm going to tell you all your faults.  Number one:
% you squeaked twice while Dinah was washing your face this
% morning.  Now you can't deny it, Kitty:  I heard you!  What that
% you say?' (pretending that the kitten was speaking.)  `Her paw
% went into your eye?  Well, that's YOUR fault, for keeping your
% eyes open---if you'd shut them tight up, it wouldn't have
% happened.  Now don't make any more excuses, but listen!  Number
% two:  you pulled Snowdrop away by the tail just as I had put down
% the saucer of milk before her!  What, you were thirsty, were you?

\chapter{\ifproject%
\ifenglish Experimentation and Results\else การทำงานและผลลัพธ์\fi
\else%
\ifenglish System Evaluation\else การประเมินระบบ\fi
\fi}

ในบทนี้จะทดสอบเกี่ยวกับการทำงานในฟังก์ชันหลักๆ

\ifproject
\include{chapters/conclusion}
\fi

\bibliography{sampleReport}

\ifproject
\normalspacing
\appendix
\include{chapters/appendix}

%% Display glossary (optional) -- need glossary option.
\ifglossary\glossarypage\fi

%% Display index (optional) -- need idx option.
\ifindex\indexpage\fi

\begin{biosketch}
\begin{center}
  \includegraphics[width=1.5in]{mugshot.jpg}
\end{center}
Your biosketch goes here. Make sure it sits inside
the \texttt{biosketch} environment.
\end{biosketch}
\fi % \ifproject
\end{document}
